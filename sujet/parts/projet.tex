\section{À propos du projet} % (fold)

\subsection{Objectifs du projet} % (fold)

Les objectifs du projet sont multiples.
Tout d'abord, l'aspect ludique du projet vous permettra d'essayer d'adopter des concepts de programmation pour un projet qui s'éloigne un peu des TPs typiques.
On peut découper le projet en plusieurs étapes, suivant les objectifs : 1) concevoir les structures de données qui représentent le jeu, 2) établir la couche de communication entre les différents joueurs et 3) établir une stratégie intelligente et cohérente.

On peut imaginer d'autres objectifs, comme par exemple l'apprentissage d'un nouveau langage par le biais de ce projet.

Vers la fin du projet, nous vous proposerons une nouvelle fonctionnalité à implémenter.
Vous devrez alors l'ajouter dans votre implémentation.
C'est pourquoi il faut dès le début adopter une conception propre, qui puisse facilement s'adapter.

% subsection Objectifs du projet (end)

\subsection{Critères} % (fold)

Le projet doit être réalisé en binôme voire trinôme.
Le langage utilisé pour implémenter le projet est libre.
Vous avez le choix entre les langages suivants : C, Python\footnote{langages pour lesquels l'enseignant se sent capable de vous guider si nécessaire}.
Libre à vous d'en proposer d'autres\footnote{que l'enseignant acceptera ou non}.

Le seul pré-requis du langage est de pouvoir créer des sockets (pour la communication des joueurs).

% subsection Critères (end)

\subsection{Notation} % (fold)

La notation finale du projet fera intervenir plusieurs aspects :

\begin{itemize}
    \item un rapport (intermédiaire et final), qui explique la conception de votre projet, les choix de programmation ainsi que les explications relatives aux stratégies, comment vous avez dû adapter votre projet lors de l'ajout de la nouvelle fonctionnalité ;
    \item une soutenance orale ;
    \item le code, qui devra être exécutable, lisible et commenté ;
    \item une compétition, en fin de semestre, où les meilleurs stratégies seront récompensées.
\end{itemize}

% subsection Notation (end)

