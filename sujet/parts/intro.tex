\section{Introduction} % (fold)
\label{sec:Introduction}

\subsection{Contexte} % (fold)
\label{sub:Contexte}

Outbreak est un jeu de simulation d'un futur où les morts-vivants\footnote{également appelés zombies, http://en.wikipedia.org/wiki/Zombie} se livrent un combat sans merci.
Dans ce futur (totalement réaliste\footnote{Si si, je vous assure}), les humains peuvent être contaminés par les germes portés par les morts-vivants et se transformer à leur tour en morts-vivants.
Ces morts-vivants ne sont toutefois pas dépourvus d'intelligence et se regroupent en équipe de zombies.

Pour s'amuser un peu, les groupes de zombies se proposent une compétition.
Cette compétition consiste à affronter plusieurs groupes de zombies dans une arène afin de déterminer quel groupe est le plus intelligent.
Dans cette arène, de nombreux humains sont mis en patûre, attendant leur fin proche.
Quelques zombies par équipe (un, voire plusieurs) sont placés dans l'arène puis la compétition commence.
Le but de chaque équipe est de contaminer le plus d'humains et d'éliminer les autres équipes.


% subsection Contexte (end)

\subsection{Votre objectif} % (fold)
\label{sub:Votre objectif}

Votre objectif est simple : diriger un groupe de zombies durant une contagion.

Outbreak se joue au tour par tour.
On vous donne à chaque tour l'état actuel de la carte et on attend de vous de prendre des décisions.
Dans notre cas, la seule action que l'on puisse faire est déplacer (ou non) un zombie.
Un zombie peut se déplacer vers le haut (Nord), vers le bas (Sud), vers la droite (Est) ou vers la gauche (Ouest).

Chaque équipe de zombies ne connait pas la totalité de la carte.
Un brouillard de guerre\footnote{http://en.wikipedia.org/wiki/Fog\_of\_war} est appliqué pour chaque joueur.
Autrement dit, un joueur ne connait que ce que ces zombies voient.

Attention, vous avez un temps limite pour prendre vos décisions à chaque tour.
La contagion se termine une fois qu'il ne reste qu'une seule équipe dans l'arène.


% subsection Votre objectif (end)

% section Introduction (end)
